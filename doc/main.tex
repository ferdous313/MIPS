\documentclass[11pt, a4paper, twoside]{IEEEtran}

\usepackage{multicol}
\setlength{\columnsep}{0.5cm}

\title
{
	Computer Architecture 2015 Fall\\
	Project 1 Report
}
\author
{
	\begin{multicols}{3}
		B03902003\\Chia-sheng Chen\\
		B03902036\\Yen-ting Liu\\
		B03902104\\Yi-ying Chao
	\end{multicols}
}
\date{\today}

\begin{document}

\maketitle

\section{Overall structure}

\subsection{Requirements}
The entire system is based on a simplified MIPS architecture. Register file contains 32 registers, with 1KBytes of instruction memory and 32Bytes of data memory.\\

Two types of hazard handling shall be implemented: data hazard and control hazard.
\begin{itemize}
\item \textbf{Data hazard} uses {\sc forwarding unit} to reduce or avoid the stall cycles, but forwarding to {\sc ID} stage isn't necessary. 
\item \textbf{Control hazard} should contain pipeline flush, and let the instructions follow by beq or j instruction to stall for 1 cycle if necessary.
\end{itemize}

% todo Not finish yet.

\subsection{Coding guideline}

\subsection{Instruction fetch}

\subsection{Instruction decode}
\subsubsection{Instruction set}
There are 10 instructions we have to implement, they are grouped into their respective types below. 

\begin{table}[h]
	\centering
	\begin{tabular}{|c|ccc|}
	\hline
    Instruction & Type & Op & Func\\
    \hline
    and & ? & 000000 & 100100 \\
    or & ? & 000000 & 100101 \\
    add & ? & 000000 & 100000 \\
    sub & ? & 000000 & 100010 \\
    mul & ? & 000000 & 011000 \\
    lw & ? & 100011 & - \\
    sw & ? & 101011 & - \\
    beq & ? & 000100 & - \\
    j & ? & 000010 & - \\
    addi & ? & 001000 & - \\
    \hline
	\end{tabular}
\end{table}

Note: Since the TAs require us to test the result using the program provided, which also limits the way how OpCodes are assigned.\\

\subsubsection{Instruction format}

\subsection{Execution}
\subsection{Memory}
\subsection{Write back}
\subsection{Pipeline}
	
\section{Modules}
	\subsection{Constants}
	\subsection{Multiplexers}
	\subsection{Registers}
		\subsubsection{Pipeline stages}
		\subsubsection{Register files}
		\subsubsection{Program counter}
	\subsection{Memories}
		\subsubsection{Instructor memory}
		\subsubsection{Data memory}
	\subsection{ALU}
		\subsubsection{Overall}
		\subsubsection{Adder}
		\subsubsection{Shift}
		\subsubsection{Comparator}
		\subsubsection{Sign extended}
	\subsection{Control units}
		\subsubsection{General control}
		\subsubsection{ALU control}
		\subsubsection{Hazard detection unit}
		\subsubsection{Forwarding control unit}
		
\section{Evaluation}
	\subsection{Timing sequence varification}
	\subsection{Test bench}
	\subsection{Fibonacci}
	
\end{document}